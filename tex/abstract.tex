% -*- coding: utf-8 -*-


\begin{zhaiyao}
这里输入中文摘要。

本文从理论和实验两方面对基于超材料的等离子激元诱导透明(PIT)新机制进行了系统的分析和研究。首先,理论分析并实验验证了一种能够在近红外波段下,实现偏振无关广角PIT现象的器件。理论上提出了一种四能级模型,很好地拟合光谱特性,并阐明了这种PIT现象的机理。证明尽管局域对称性破缺是实现PIT现象的必要条件,但是这并不意味着PIT器件无法实现对称。这种新型的局域非对称、整体旋转对称的微纳结构设计思路将对光学调控器件的发展提供有益的指导。然后,设计并模拟分析了一种在近红外谱段,具有动态调控性的PIT平面杂化超材料,此设计将近场耦合效应引入动态超材料领域。我们优化了调控性材料的嵌入方式,温敏二氧化钒薄膜被填入狭缝天线的镂空处,作为等离子激元系统的组成部分,这将极大提高PIT的动态调制深度。此外,我们利用四能级模型,定量分析了这种动态调控PIT器件。这种新型杂化超材料将为动态PIT器件的设计提供有益的指导。
\end{zhaiyao}




\begin{guanjianci}
毕业论文;模板
\end{guanjianci}



\begin{abstract}
This is the abstract.
\end{abstract}



\begin{keywords}
Thesis; template
\end{keywords} 