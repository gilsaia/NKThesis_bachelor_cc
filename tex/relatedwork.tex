% -*- coding: utf-8 -*-


\chapter{NKThesis 相关说明} 
\label{chpt:relatedwork}

\section{系统要求}

模板仅在 TeXLive 2020 下测试通过。对于其它 TeX 发行版可能需要做个别修改。

另外,本模板用到的minted宏包需要python环境并在python环境安装python包Pygement。

\section{NKThesis 使用说明}

本模板可以使用以下方式编译:
\begin{enumerate}
 \item \XeLaTeX [推荐]
\end{enumerate}

例如,
\begin{verbatim}
         xelatex main
         biber main      % 处理参考文献
         xelatex main   % 连续编译两遍以生成正确的文献引用。
\end{verbatim}





本模板用到 宋体、楷体、仿宋、黑体四种字体,通过fontspec宏包从系统字体中获取.
对于 Linux/Mac 下的环境, 不确定宏包是否仍然有效,可能需要自行查找设置。


我们建议您使用\XeLaTeX\ 编译。与前两种方法相比,\XeLaTeX\  编译长文档的速度更快,
编译一篇一百多页的论文只需几秒的时间(SL9400 @ 1.86GHz)。

在改变编译方式前应先删除 *.toc 和 *.aux 文件,
因为不同编译方式产生的辅助文件格式可能并不相同。


\section{引用章节号}
\label{sec:ex:A}

引用章节号请参考如下格式: \ref{chpt:relatedwork}\ref{sec:ex:A}.


\section{中英文间隔}

使用 \XeLaTeX\ 编译时,会自动在中英文转换时添加必要的空格。 使用 [PDF]\LaTeX\
编译时仅忽略中文之间的空格,而中英文之间的空格予以保留。
因此,不管何种编译方式,您都不需要在中英文间添加 $\tilde{}$ 以获得额外的空格。例如,

这是 English 中文 $x=y$ 测试

这是English中文$x=y$测试

可以看出,以上两行用 \XeLaTeX\ 编译的结果是相同的。


\section{NKThesis 预调用的宏包}

NKThesis 已经调用以下宏包,您无须重新调用。

\begin{table}[htbp]
  \caption{NKThesis 预调用的宏包}
  \zihao{5}
  \begin{center}
    \begin{tabular}{lc}
      \toprule
      编译方式 & 调用的宏包\\ 
      \midrule
      \XeLaTeX & ctex,graphicx,amssymb,amsthm,minted,xcolor,booktabs,caption\\
        & enumitem,fontspec,titlesec,titletoc,fancyhdr,hypreref,pgf,tikz,biblatex \\ 
      \bottomrule
      \end{tabular}
  \end{center}
\end{table}


\section{图表}
\label{sec:relatedwork:table}

插图的例子:

\begin{figure}[hbtp]
  \centering
  \includegraphics[viewport=0 0 2984 969,width=40mm]{nankaidaxue.pdf}  
  \caption{\label{fig:nku}南开大学}
\end{figure}

\section{字体}

一般情况下, 您不需要显式地设置字体. 如果确实需要, 请使用以下命令

\begin{verbatim}
宋体:  \songti
黑体:  \heiti
楷体:  \kaishu
仿宋:  \fangsong
加粗:  \bfseries
\end{verbatim}

而对于字号调整\footnote{大部分情况下也并不需要}可以通过ctex定义的命令
\begin{verbatim}
  \zihao{3}
\end{verbatim}
来设置,3表示三号,-3表示小三等等,在大部分情况下这都是可以使用的\footnotemark。
\footnotetext{这是脚注的第二种写法}


\section{参考文献} \label{sec:relatedwork:ref}
参考文献引用:
\cite{ChenCheChen2001,Nadkarni-1992,Hua-Wang-1973}
\cite{ZhuKeZhen,Huo,Example}\cite{JiangXiZhou,Timoshenko,Zhang-Wang,Ding,GB6447-86}
\cite[Theorem 2.1]{ZhuKeZhen}

\subsection{录入参考文献}

本模板采用 biblatex 宏包管理参考文献。如果你对此不熟悉,可以
\begin{enumerate}
\item 参考宏包使用说明,或者
\item 手工排版参考文献,然后参考 nkthesis.bib 最后 3 条的格式录入。
\end{enumerate}



\section{一些建议}
\subsection{关于分数的写法}
\label{sec:relatedwork:equation}


\LaTeX 提供宏命令\verb+\frac+, 用以打印分数. 为使得版面整齐, 该命令的使用应遵循以下原则:

\begin{enumerate}
\item 仅在分行表达式中使用,
\item 不嵌套使用,
\item 不在上下标中使用.
\end{enumerate}

也就是说, 行内表达式和上下标中出现分数时一律用 $a/b$表示, 如
$(x+2)/((3x^2+4)(7+y))$. 下面是居中表达式:

\[
 x^2 = y^{1/2} +3.
\]

多行表达式: 尽量在加、减、乘、等号前换行. 在乘号前换行时,
下一行首用 \string\times:
\def\iint{\mathop{\int\!\!\!\int}}\def\calG{\mathcal G}
\begin{eqnarray}
&&\left|(W_{\psi_1}f)(a,b)-(W_{\psi_1}f)(a_j,b_{j,k})\right|^{2}\nonumber\\
&=&\frac{1}{C^{2}_{\varphi}}\Bigg|\iint_{\calG} (W_{\varphi}f)(s,t) \nonumber\\
&&\qquad\times \Bigg( (W_{\psi_1}\varphi)\left(\frac{a}{s},
\frac{b-t}{s}\right)
     -(W_{\psi_1}\varphi)\left(\frac{a_{j}}{s}, \frac{b_{j,k}-t}{s}\right)\Bigg)
  \frac{dsdt}{s^{d+1}}\Bigg|^2 \nonumber\\
&\le& \frac{1}{C^2_{\varphi}} \iint_{\calG} |(W_{\varphi}f)(s,t)|^2 \nonumber\\
&&\qquad \times\left| (W_{\psi_1}\varphi)\left(\frac{a}{s},
\frac{b-t}{s}\right)
    -(W_{\psi_1}\varphi)\left(\frac{a_{j}}{s}, \frac{b_{j,k}-t}{s}\right)\right|
   \frac{dsdt}{s^{d+1}}  \nonumber\\
&&\qquad \times   \iint_{\calG}\!
 \left|(W_{\psi_1}\varphi)\left(\frac{a}{s}, \frac{b-t}{s}\right)
    -(W_{\psi_1}\varphi)\left(\frac{a_{j}}{s}, \frac{b_{j,k}-t}{s}\right)\right|
 \frac{ ds dt}{s^{d+1}} \nonumber\\
&=& \frac{1}{C^2_{\varphi}} ....  \label{eq:a0}
\end{eqnarray}


\subsection{标点}
就本科生毕设而言,我们应使用默认的全角标点。

科技文献中一般用半角标点, 请参考《中国科学》发表的论文.

如果使用全角标点, 可以使用
\begin{verbatim}
  \punctstyle{<style>}
\end{verbatim}
选择标点样式, 有效值为
\begin{verbatim}
  quanjiao (所有标点符号占一个汉字宽度,
            相邻两个标点占一个半汉字宽度)
  banjiao  (所有标点符号占半个汉字宽度)
  hangmobanjiao (所有标点符号占一个汉字宽度,行末行首半角)
  kaiming  (句号、叹号、问号占一个汉字宽度,其他标点占半个汉字宽度)
\end{verbatim}
缺省为全角式。注意:不论选择哪种样式,都提供行末对齐(margin kerning)功能。



\begin{Theorem} \label{thm:latex}
\LaTeX 的输出是最完美的.
\end{Theorem}

先证明一个引理

\begin{Lemma} \label{thm:tex}
\TeX 文件在不同操作系统下的排版结果完全一致.
\end{Lemma}

\begin{Definition}
\cite{Timoshenko}
这是定义.
\end{Definition}


\begin{Proof}
定理\ref{thm:latex}显然是错的.
\end{Proof}

单个带编号的表达式
\begin{equation}\label{eq:a1}
x=y+z
\end{equation}

单个不带编号的表达式
\[
y=x-z.
\]

不带编号的多行表达式
\begin{eqnarray*}
x&=&y+z \\
 &=&z-s\\
 &<& 3. \\
 && \mbox{一些注释}
\end{eqnarray*}

带编号的多行表达式
\begin{eqnarray}
 x&=& y-z, \label{eq:aa1}\\
 y&=& x+z, \nonumber \\
 z&=&y-x. \label{eq:aa2}
\end{eqnarray}



引用:   定理\ref{thm:latex}的推论是什么呢?
方程式编号:  由(\ref{eq:a1})(\ref{eq:aa2})式.

\subsection{列举环境:  enumerate}

环境 enumerate 用来输出有序列表 多级列表的编号统一如下示例

  \begin{enumerate}
    \item This is an example.
    \item This is an example.
      \begin{enumerate}
      \item This is an example.
      \item This is an example.
    \end{enumerate}
  \end{enumerate}
  \begin{enumerate}
    \item This is an example.
    \item This is an example.
      \begin{enumerate}
      \item This is an example.
      \begin{enumerate}
        \item This is an example.
      \end{enumerate}
      \item This is an example.
    \end{enumerate}
  \end{enumerate}
